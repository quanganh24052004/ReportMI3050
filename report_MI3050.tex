% === tHIẾT LẬP TÀI LIỆU CƠ BẢN === %
\documentclass[oneside, a4paper]{book} % Kiểu tài liệu dạng sách, in 1 mặt, khổ A4
\usepackage[left=3.5cm,right=2cm,top=3.5cm,bottom=3cm]{geometry} % Thiết lập lề trang
\renewcommand{\baselinestretch}{1.5} % Giãn dòng 1.5
\usepackage{indentfirst} % Thụt đầu dòng đoạn đầu tiên

% === GÓI HỖ TRỢ HIỂN THỊ NỘI DUNG === %
\usepackage{graphicx}      % Hỗ trợ chèn hình ảnh
\usepackage{wrapfig}       % Chèn hình ảnh bao quanh văn bản
\usepackage{float}         % Quản lý vị trí hình/table
\usepackage{longtable}     % Hỗ trợ bảng dài qua nhiều trang
\usepackage{tabularx}      % Bảng co giãn theo chiều ngang
\usepackage{multirow}      % Ô bảng có thể chiếm nhiều dòng
\usepackage{fancybox}      % Khung hộp đẹp
\usepackage{caption}       % Tùy chỉnh chú thích hình, bảng
\usepackage{listings}      % Hiển thị mã nguồn
\usepackage{xcolor}        % Hỗ trợ màu sắc
\usepackage{tikz, pgf}     % Vẽ hình bằng TikZ
\usetikzlibrary{shapes, patterns, arrows.meta, calc} % Thư viện bổ trợ cho TikZ
\usepackage{pdfpages}      % Chèn file PDF vào tài liệu
\usepackage{picinpar,floatflt} % Chèn hình trong đoạn văn
\usepackage{enumitem}      % Tùy chỉnh danh sách liệt kê
\usepackage{hyperref}      % Tạo liên kết (mục lục, tham chiếu chéo...)
\usepackage{microtype}     % Tối ưu dàn chữ
\usepackage{anyfontsize}   % Tùy chỉnh kích thước font tự do

% === ĐỊNH DẠNG MÃ NGUỒN === %
\definecolor{codegreen}{rgb}{0,0.6,0}
\definecolor{codegray}{rgb}{0.5,0.5,0.5}
\definecolor{codepurple}{rgb}{0.58,0,0.82}
\definecolor{backcolour}{rgb}{0.95,0.95,0.92}

\lstdefinestyle{mystyle}{
	backgroundcolor=\color{backcolour},
	commentstyle=\color{codegreen},
	keywordstyle=\color{magenta},
	numberstyle=\tiny\color{codegray},
	stringstyle=\color{codepurple},
	basicstyle=\ttfamily\normalsize,
	breakatwhitespace=false,
	breaklines=true,
	captionpos=b,
	keepspaces=true,
	numbers=left,
	numbersep=5pt,
	showspaces=false,
	showstringspaces=false,
	showtabs=false,
	tabsize=2
}
\lstset{style=mystyle}

% === TOÁN HỌC VÀ KÝ HIỆU === %
\usepackage{amsmath,amssymb,latexsym,amscd,amsxtra,amsthm} % Các gói toán học
\usepackage{mathtools}       % Nâng cao cho amsmath
\usepackage{pb-diagram}      % Vẽ sơ đồ toán học
\usepackage{aligned-overset} % Dấu overset có kèm căn chỉnh
%\usepackage{commath}        % (đã comment) Hỗ trợ đạo hàm, tích phân đẹp

% === ĐỊNH NGHĨA MÔI TRƯỜNG ĐỊNH LÝ, VÍ DỤ, ĐỊNH NGHĨA === %
\theoremstyle{plain}
\newtheorem{bd}{Bổ đề}[chapter]
\newtheorem{md}{Mệnh đề}[chapter]
\newtheorem{hq}{Hệ quả}[chapter]
\newtheorem{dl}{\bfseries Định lý}[chapter]
\newtheorem{tc}{\bfseries Tính chất}[chapter]

\theoremstyle{definition}
\newtheorem{dn}{Định nghĩa}[chapter]
\newtheorem{bt}{Bài toán}[chapter]
\newtheorem{vd}{\bfseries Ví dụ}[chapter]
\newtheorem{algorithm}{\bfseries Thuật toán}[chapter]
\newtheorem{assumption}{\bfseries Giả thiết}[chapter]

\theoremstyle{remark}
\newtheorem{kh}{ký hiệu}[chapter]
\newtheorem{nx}{Nhận xét}[chapter]
\newtheorem{ly}{Lưu ý}[chapter]

\renewcommand{\thekh}{\textit{\thechapter.\arabic{kh}}}
\renewcommand{\thenx}{\textit{\thechapter.\arabic{nx}}}
\renewcommand{\thely}{\textit{\thechapter.\arabic{ly}}}

\newtheorem{cau}{Câu}[] % Không đánh số theo chương

% Môi trường định lý được đặt tên riêng
\newtheorem*{namedthm}{\namedthmname}
\newcounter{namedthm}
\makeatletter
\newenvironment{named}[1]
  {\def\namedthmname{#1}%
   \refstepcounter{namedthm}%
   \namedthm\def\@currentlabel{#1}}
  {\endnamedthm}
\makeatother

% Định nghĩa môi trường chứng minh
\newenvironment{cm}{\chm}{\eproof}
\newenvironment{mproof}{\paragraph{Chứng minh:}}{\hfill$\square$}
\newenvironment{myproof}[2] {\paragraph{Proof of {#1} {#2} :}}{\hfill$\square$}

% Cú pháp viết ký hiệu dễ dùng
\DeclarePairedDelimiter{\pro}{\langle}{\rangle}
\DeclarePairedDelimiter{\norm}{\lVert}{\rVert}
\DeclarePairedDelimiter{\abs}{\lvert}{\rvert}

% === GIAO DIỆN TRANG: LỀ, HEADER, ĐÁNH SỐ === %
% Các thông số lề chi tiết
\setlength{\oddsidemargin}{0.3cm}
\setlength{\topmargin}{-1cm}
\setlength{\headsep}{0.5cm}
\textwidth=15.5cm
\textheight=24.5cm

% Header & footer
\usepackage{fancyhdr}
\fancyhf{}
\fancyhead[C]{\thepage}
\pagestyle{fancy}
\fancypagestyle{plain}{%
  \fancyhf{}
  \fancyhead[C]{\thepage}
  \renewcommand{\headrulewidth}{0pt}
}

% Bỏ dấu chấm sau tiêu đề định lý
\usepackage{xpatch}
\makeatletter
\AtBeginDocument{\xpatchcmd{\@thm}{\thm@headpunct{.}}{\thm@headpunct{}}{}{}}
\makeatother

% Cấu hình mục lục chương
\usepackage{titletoc}
\titlecontents{chapter}
  [0pt]
  {\bfseries}
  {\chaptername\ \thecontentslabel\enskip}
  {}
  {\hfill\contentspage}

% Đặt lại kiểu đánh số chương, mục
\renewcommand{\thechapter}{\arabic{chapter}}
\renewcommand{\thesection}{\arabic{chapter}.\arabic{section}}

% === HỖ TRỢ TIẾNG VIỆT VÀ MỤC LỤC MINI === %
\usepackage[utf8]{vietnam}            % Gói gõ tiếng Việt
\usepackage[tight,vietnam]{minitoc}   % Mục lục mini đầu mỗi chương
\renewcommand{\bibname}{Tài liệu tham khảo} % Đổi tên thư mục tài liệu tham khảo

% === CÁC ĐỊNH NGHĨA MACRO KHÁC === %
\newcommand{\eproof}{\hfill $\square$}
\newcommand{\chm}{{\bfseries Chứng minh.}}

\begin{document}
	
\thispagestyle{empty}
%\thispagestyle{headings}
\setcounter{page}{1}%
\pagenumbering{roman}
%\addcontentsline{toc}{chapter}{{Trang bìa phụ}}
%\newgeometry{top=2.0cm,bottom=3.0cm,left=3.5cm,right=2.8cm}
%\begin{titlepage}

%\setlength{\fboxrule}{1pt}
%\thisfancypage{\setlength{\fboxsep}{2pt}\setlength{\fboxrule}{2pt}\doublebox}{} 
%\setlength{\fboxrule}{1pt}
%\thisfancypage{\setlength{\fboxsep}{0pt}\setlength{\shadowsize}{0pt}\doublebox}{}
\pagenumbering{gobble}
\newgeometry{left=2.8cm,right=1.8cm,top=2.5cm,bottom=2cm}

% CoverPage %


\begin{center}
    \Large
    \vspace{-1cm}
    {\bfseries ĐẠI HỌC BÁCH KHOA HÀ NỘI}\\
    {\bfseries KHOA TOÁN - TIN}
\end{center}


\vspace{4.5cm}

\begin{center}
	\fontsize{20pt}{16pt}\selectfont 	
	{\bfseries CÁC PHƯƠNG PHÁP TỐI ƯU BÀI TOÁN VẬN TẢI VÀ PHƯƠNG PHÁP THẾ VỊ}

    \vspace{1cm}
    
    \fontsize{22pt}{16pt}\selectfont
	{\bfseries }

    \vspace{1cm}
    \fontsize{16pt}{16pt}\selectfont{\textbf{Nhóm sinh viên thực hiện:}}
    \begin{table}[h]
        \centering
        {\fontsize{16pt}{16pt}\selectfont % <-- mở ngoặc để nhóm font size
        \begin{tabular}{clc}
        \textbf{MSSV} & \multicolumn{1}{c}{\textbf{Họ và Tên}} & \textbf{Lớp} \\
        20227076      & Nguyễn Thanh An                        &              \\
        20227079      & Nghiêm Hoàng Anh                       &              \\
        20227035      & Nguyễn Quang Anh                       &              \\
        20227080      & Nguyễn Thị Thảo Anh                    &              \\
        20227081      & Phạm Đức Anh                           &              \\
        20227082      & Phạm Tuấn Anh                          &              \\
        20227083      & Lê Gia Bảo                             &              \\
        20227085      & Ngô Trọng Bảo                          &              \\
        20227086      & Trần Ngọc Bảo                          &              \\
        20227036      & Đỗ Văn Bình                            &             
        \end{tabular}
        } % <-- đóng nhóm font size
    \end{table}
    \vspace{0.5cm}
	
    %\fontsize{13pt}{16pt}\selectfont
	%{{Chuyên ngành:} \fontsize{15pt}{16pt}\selectfont \textbf {Toán - Tin}}
%
    %\fontsize{13pt}{16pt}\selectfont
	%{{Định hướng:} \fontsize{15pt}{16pt}\selectfont \textbf{Toán ứng dụng}}
\end{center}

%\vspace{2cm}

%\fontsize{14pt}{16pt}\selectfont
%{
%    \begin{tabular}{ l l }
%        \hspace{-1.5cm}{Giảng viên hướng dẫn:}& \fontsize{16pt}{16pt}\selectfont \textbf{PGS.TS. Trịnh Ngọc Hải} \quad \newcommand\tab[1][0.3cm]{\hspace*{#1}}\tab $\underset{\text{Chữ ký của GVHD}}{\rule{3.5cm}{0.5pt}}$ \\
%    \end{tabular}
%}

\vspace{2cm}

\begin{center}
    \LARGE \textbf{HÀ NỘI -- 2025}
\end{center}

\restoregeometry



% %
\newpage

% Observe %

%\newgeometry{left=2.2cm,right=2.8cm,top=2cm,bottom=2cm}
%\pagenumbering{roman}
%\setcounter{page}{1}
\thisfancypage{
	\setlength{\fboxsep}{0.5cm}
	\fbox}{}
 
\fontsize{13pt}{16pt}\selectfont
\bigbreak
\begin{center}
	{\bf NHẬN XÉT CỦA GIẢNG VIÊN HƯỚNG DẪN}
\end{center}
%\bigbreak

\fontsize{12pt}{14pt}\selectfont
\begin{enumerate}
 \item [{\bf 1.}]{\bf Mục tiêu và nội dung của đồ án}
 \begin{enumerate}
 \item Mục tiêu: Đề tài nghiên cứu một phương pháp lặp hiện với cỡ bước tự tích nghi xấp xỉ nghiệm một lớp bài toán chấp nhận tách trong không gian Hilbert thực và áp dụng cho bài toán khôi phục ảnh.
 \item Nội dung: Trình bày một số kiến thức cơ bản của không gian Hilbert thực, khái niệm và ví dụ về bài toán chấp nhận tách; trình bày phương pháp lặp hiện giải bài toán; chứng minh sự hội tụ mạnh của dãy lặp và trình bày áp dụng cho bài toán khôi phục ảnh.
 \end{enumerate}
    
\item [{\bf 2.}] {\bf Kết quả đạt được} 
 \begin{enumerate}
\item ....
\end{enumerate}
\item [{\bf 3.}]{\bf Ý thức làm việc của sinh viên:}
\begin{enumerate}
\item ...
\end{enumerate}
\end{enumerate}

\vspace{5mm}
\hspace{0.5\textwidth}
\begin{minipage}{0.5\textwidth}
\noindent\begin{center}
\textit{Hà Nội, ngày ... tháng ...  năm 2025} \\
Giảng viên hướng dẫn\\ \vspace{0.6cm}
\vspace{1.5cm}
\textbf{PGS.TS. Nguyễn Thị Thu Thủy}
\end{center}	
\end{minipage}
\restoregeometry


% %
\newpage

% Thanks %

\chapter*{Lời cảm ơn}
%\addcontentsline{toc}{chapter}{\bfseries Lời cảm ơn}

Báo cáo này ...

\vspace{8mm}


% %
\newpage

% abbreviate %

\chapter*{Tóm tắt nội dung báo cáo}
%\addcontentsline{toc}{chapter}{\bfseries Tóm tắt nội dung báo cáo}

Mục tiêu ...

\vspace{-3mm}
\hspace{0.5\textwidth}
\begin{minipage}{0.5\textwidth}
	\noindent\begin{center}
		\vspace{1cm}
		\textit{Hà Nội, ngày ... tháng ... năm 2025} \\
		Tác giả đồ án\\ \vspace{1.5cm}
		\textbf{Nhóm sinh viên thực hiện}
	\end{center}	
\end{minipage}


% %

\tableofcontents % Là xuất hiện mục lục.
\pagestyle{myheadings}

\newpage

%\addcontentsline{toc}{chapter}{Danh sách bảng}
\thispagestyle{empty}
\pagenumbering{arabic}
\setcounter{page}{1}

% tableOfSAA %
\chapter*{Bảng ký hiệu và chữ viết tắt}
%\addcontentsline{toc}{chapter}{\bfseries Bảng ký hiệu và chữ viết tắt}
\begin{tabular}
    {@{\hspace{0.6cm}} l @{\hspace{1.8cm}}p{11.5cm}l}
    $\mathcal{H}$ & không gian Hilbert thực\\

\end{tabular}


% Danh sách bảng %
\newpage
\listoftables
%\addcontentsline{toc}{chapter}{\bfseries Danh sách bảng}
\thispagestyle{empty}
%

% Danh sách hình ảnh %
\newpage
\renewcommand{\listfigurename}{Danh sách hình ảnh}
\listoffigures
%\addcontentsline{toc}{chapter}{\bfseries Danh sách hình ảnh}
 %


%\chapter*{Mở đầu}
%\addcontentsline{toc}{chapter}{\bfseries Mở đầu}
%\large
\setlist{nosep}

\clearpage

\chapter*{Mở đầu}
\addcontentsline{toc}{chapter}{Mở đầu}
Bài toán vận tải là một mô hình quan trọng trong lĩnh vực \textbf{quy hoạch tuyến tính}, có ứng dụng rộng rãi trong \textbf{quản lý logistics}, \textbf{tối ưu hóa chuỗi cung ứng}, \textbf{phân phối tài nguyên} và nhiều lĩnh vực khác. Mục tiêu của bài toán là tìm phương án vận chuyển tối ưu từ các nguồn đến các điểm tiêu thụ sao cho tổng chi phí là nhỏ nhất và thỏa mãn các ràng buộc cung – cầu.

Báo cáo này tập trung:
\begin{itemize}
    \item Tổng quan lý thuyết nền tảng và mô hình bài toán.
    \item Phân tích các phương pháp giải cơ bản và mở rộng.
    \item Trình bày chi tiết các kỹ thuật khởi tạo phương án xuất phát hiệu quả.
\end{itemize}

\chapter{Tổng quan và cơ sở lý thuyết}
\section{Giới thiệu \& Mô hình toán học}
Bài toán vận tải (Transportation Problem) là dạng đặc biệt của quy hoạch tuyến tính, mô tả việc phân phối hàng hóa từ $m$ nguồn cung ($A_i$) đến $n$ điểm cầu ($B_j$). Ở dạng chuẩn, biến quyết định $x_{ij}$ là lượng hàng từ $A_i$ đến $B_j$, với hàm mục tiêu:
\[
\min \sum_{i=1}^m \sum_{j=1}^n c_{ij} x_{ij}
\]
và các ràng buộc:
\[
\sum_j x_{ij} = a_i,\quad \sum_i x_{ij} = b_j,\quad x_{ij} \geq 0
\]

\section{Điều kiện tồn tại nghiệm}
Để bài toán có nghiệm khả thi và tối ưu, thường yêu cầu \textbf{cân bằng thu-phát}: $\sum_i a_i = \sum_j b_j$. Nếu không cân bằng, có thể bổ sung điểm phát hoặc nhận giả để cân bằng. Khi cân bằng, luôn tồn tại phương án cơ bản tối ưu.

\section{Bảng vận tải và chu trình}
Bài toán được biểu diễn dưới dạng bảng chi phí $m \times n$. Mỗi phương án cơ bản không suy biến ứng với $m+n-1$ ô chọn (không chứa chu trình). \textbf{Chu trình} là dãy các ô trong bảng nối tiếp theo hàng/cột sao cho ô đầu và ô cuối liền kề. Chu trình được sử dụng để điều chỉnh lưu lượng và kiểm tra cải tiến nghiệm.

\clearpage

\chapter{Phương pháp giải bài toán vận tải}
\section{Phương pháp thế vị}
Đây là thuật toán tối ưu hóa dựa trên ý tưởng kiểm tra từng ô không thuộc cơ sở bằng cách xây dựng chu trình khép kín, tính hệ số cải tiến $\Delta_{ij}$ để xem liệu giảm chi phí tổng có thể thực hiện được hay không. Phương pháp thế vị tương đương với bài toán lõi kép của quy hoạch tuyến tính.

\section{Các bài toán mở rộng}
\begin{itemize}
    \item \textbf{Vận tải không cân bằng}: Tổng cung $\neq$ tổng cầu. Thêm điểm giả để cân bằng.
    \item \textbf{Lập kho trung chuyển}: Cho phép hàng đi qua các điểm trung gian.
    \item \textbf{Ô cấm}: Một số ô không cho phép vận chuyển.
    \item \textbf{Ràng buộc bất đẳng thức}: Biến ràng buộc cung/cầu thành bất đẳng thức.
    \item \textbf{Bài toán phân công (Hungary)}: Trường hợp $a_i = b_j = 1$, giải bằng phương pháp Hungary.
    \item \textbf{Bài toán chuyển hàng}: Cho phép chuyển qua trạm trung gian, giải qua quy hoạch tuyến tính.
\end{itemize}

\clearpage

\chapter{Các phương pháp tìm phương án xuất phát}
\section{Mục tiêu khởi tạo}
Tìm phương án cơ sở ban đầu khả thi để bắt đầu quá trình tối ưu.

\section{Các phương pháp khởi tạo}
\begin{itemize}
    \item \textbf{Phương pháp Góc Tây Bắc}
    \item \textbf{Phương pháp Chi phí thấp nhất}
    \item \textbf{Phương pháp Vogel (VAM)}
\end{itemize}

\section{So sánh \& minh họa}
So sánh các phương pháp về số phép toán, độ gần tối ưu, và hiệu quả. Kèm ví dụ minh họa chi tiết.


\clearpage

\chapter{Kết luận}
Bài toán vận tải là mô hình tối ưu hóa cơ bản và hiệu quả trong logistics. Báo cáo đã trình bày tổng quan, phương pháp giải, và khởi tạo bài toán. Các hướng mở bao gồm bài toán động, bất định, hoặc dùng thuật toán di truyền.

%\chapter*{Kết luận}                         % Chương 3
%\addcontentsline{toc}{chapter}{Kết luận}
%\setcounter{section}{0}
%\renewcommand*{\thesection}{\the\value{section}}


%\section*{Kết quả đạt được} 

%\begin{itemize}
%    \item ...
%\end{itemize}
%
%\vspace{5mm}
%\section*{Kỹ năng đạt được}
%\begin{itemize}
%    \item Nghiên cứu khoa học, viết bài báo nghiên cứu và đăng trên tạp chí khoa học.
%    \item Tìm kiếm tài liệu chuyên ngành và tổng hợp các kiến thức tìm hiểu để áp dụng vào nội dung đồ án.
%    \item Trình bày đồ án một cách logic, chặt chẽ theo chuẩn khoa học, chế bản đồ án bằng \LaTeX.
%    \item Xây dựng các ví dụ minh họa phù hợp và thiết kế các chương trình thực thi thuật toán bằng Python. 
%    \item Áp dụng phương pháp đề xuất để giải quyết một số bài toán trong kinh tế và kỹ thuật.
%\end{itemize}
%
%
%\section*{Hướng phát triển của đồ án trong tương lai}
%
%\begin{itemize}
%    \item ...
%\end{itemize}


\begin{thebibliography}{99}\rm
    \addcontentsline{toc}{chapter}{\bfseries Tài liệu tham khảo}
    
    \section*{Tiếng Việt}
    
    \bibitem{Kim} Nguyễn Thị Bạch Kim, \textit{Các phương pháp tối ưu – Lý thuyết và thuật toán}, , NXB Đại học Quốc gia TP.HCM, 2020.
    
    \section*{Tiếng Anh}
    
    \bibitem{abc} 
    \end{thebibliography}
    
    % Lưu số thứ tự cuối cùng của tài liệu tham khảo
    \newcounter{lastbibitem}
    \setcounter{lastbibitem}{\value{enumiv}}
    
    %\renewcommand{\bibname}{Công trình nghiên cứu liên quan}
    %\begin{thebibliography}{99}\rm
    %\addcontentsline{toc}{chapter}{\bfseries Công trình nghiên cứu liên quan}
    %% Bắt đầu tiếp nối số thứ tự
    %\setcounter{enumiv}{\value{lastbibitem}}
    %
    %\section*{Tiếng Anh}
    %\bibitem{abc} ...
    %\end{thebibliography}
    %
    %\textit{Tóm tắt nội dung nghiên cứu}: ...
    
    
    %\includepdf[pages=-, scale=0.95, pagecommand={\thispagestyle{fancy}}]{paper.pdf} % Chèn toàn bộ file pdf
    

\end{document}