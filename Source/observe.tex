\newgeometry{left=2.2cm,right=2.8cm,top=2cm,bottom=2cm}
%\pagenumbering{roman}
%\setcounter{page}{1}
\thisfancypage{
	\setlength{\fboxsep}{0.5cm}
	\fbox}{}
 
\fontsize{13pt}{16pt}\selectfont
\bigbreak
\begin{center}
	{\bf NHẬN XÉT CỦA GIẢNG VIÊN HƯỚNG DẪN}
\end{center}
%\bigbreak

\fontsize{12pt}{14pt}\selectfont
\begin{enumerate}
 \item [{\bf 1.}]{\bf Mục tiêu và nội dung của đồ án}
 \begin{enumerate}
 \item Mục tiêu: Đề tài nghiên cứu một phương pháp lặp hiện với cỡ bước tự tích nghi xấp xỉ nghiệm một lớp bài toán chấp nhận tách trong không gian Hilbert thực và áp dụng cho bài toán khôi phục ảnh.
 \item Nội dung: Trình bày một số kiến thức cơ bản của không gian Hilbert thực, khái niệm và ví dụ về bài toán chấp nhận tách; trình bày phương pháp lặp hiện giải bài toán; chứng minh sự hội tụ mạnh của dãy lặp và trình bày áp dụng cho bài toán khôi phục ảnh.
 \end{enumerate}
    
\item [{\bf 2.}] {\bf Kết quả đạt được} 
 \begin{enumerate}
\item ....
\end{enumerate}
\item [{\bf 3.}]{\bf Ý thức làm việc của sinh viên:}
\begin{enumerate}
\item ...
\end{enumerate}
\end{enumerate}

\vspace{5mm}
\hspace{0.5\textwidth}
\begin{minipage}{0.5\textwidth}
\noindent\begin{center}
\textit{Hà Nội, ngày ... tháng ...  năm 2025} \\
Giảng viên hướng dẫn\\ \vspace{0.6cm}
\vspace{1.5cm}
\textbf{PGS.TS. Nguyễn Thị Thu Thủy}
\end{center}	
\end{minipage}
\restoregeometry
